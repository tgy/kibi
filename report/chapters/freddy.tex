\chapter{Freddy -- \emph{Segmentation}}

Freddy is our segmentation module. It uses OCSFML.
To detect lines as today, we first assume that all the previous preprocessing steps were done perfectly, which means no artefacts, good rotation, binarization... For now, we can only get text, in one column, and no images. We will work on those situations, and Freddy will be working as expected on time.

Freddy est notre module de segmentation, il utilise OCSFML.

\section{L'ancienne méthode, ses points faibles}

Au moment de la première soutenance, nous utilisions un algorithme très simple, et moyennement efficace: nous faisions un histogramme horizontal, un tableau de taille la hauteur en pixels de l´image. Nous le remplissions avec le nombre de pixels noirs à la ligne correspondante. Cette opération nous permettait de savoir à quel moment une nouvelle ligne ou un nouveau paragraphe apparaissent.\\
Une fois ce découpage effectué, il nous suffisait de répéter cette opération horizontalement sur chacune des lignes repérées, afin de détecter les mots et caractères composant chacune des lignes.\\
Cependant, nous avons été confrontés à plusieurs problèmes, qui nous ont amenés à chercher une autre méthode plus performante, que nous verrons en détail dans la prochaine section. En effet, avec cette méthode, les caractères avaient tendance à être coupés, ou bien collés à d´autres. Il en était de même pour les lignes, qui avaient tendance à fusionner. Cette technique nécessitait également d´avoir une image sortie du prétraitement parfaite, la moindre impureté ou la moindre ligne, image ruinant totalement le résultat de l´algorithme.

\section{Le RLSA}

Pour détecter les paragraphes, lignes, mots ou les caractères, nous utilisons donc maintenant l'algorithme Run Lenght Smooth Algorithm (RLSA) modifié par nos soins, pour couvrir au mieux nos besoins. Le principe est simple: on regarde horizontalement et verticalement quels sont les pixels noirs adjacents, c´est à dire quels sont les pixels noirs séparés de moins de n pixels noirs. Il en résultera donc deux matrices de bouléens aux dimensions de l´image. Dans ces matrices représentant la couleur noire ou blanche des pixels de l´image, nous aurons changé tous les pixels blancs entre deux pixels noirs adjacents. Les images représentées par ces matrices ressemblent à l´image d´entrée, mais comme si l´encre avait bavée horizontalement ou verticalement.

\subsection{Optimisations}

\begin{center}
\end{center}

\begin{center}
\end{center}

\section{Line and character detection}

\subsection{Making horizontal/vertical histogram}

To detect lines, we need to make histograms, containing the number of black pixels in every line/column of line of pixel of the provided image.We put everything in lists.

\subsection{Delimit lines/characters}

When the number of black pixel changes a lot, we decide to create or to end a line/to create or end a character. We put it in lists.

\subsection{Delimit lines/characters}

When the number of black pixel changes a lot, we decide to create or to end a line/to create or end a character. We put it in lists
