\chapter{Le manuel d´utilisation}

Nous allons détailler les différentes fonctionnalités de notre OCR.
\begin{itemize}
  \item{Une fois connecté à l´interface web, il est possible de choisir une image, puis de lancer le traitement. Cliquez sur le bouton ¨Choose an image to process¨, et sélectionnez votre image.}
  \item{Une fois terminé, cliquez sur ¨Start process¨. Le temps pris par chacune des étapes est affichée en secondes.}
  \item{Le traitement terminé, cliquez sur ¨See the result¨ pour voir le résultat final de notre OCR.}
  \item{Une page affichant le texte reconnu est affichée: les mots non présents dans le dictionnaire de robert sont surlignés en rouge. Un clic affichera sous forme d´un volet déroulant les éventuelles corrections proposées par robert. Un clic sur un de ces mots l´insèrera à la place du mot qu´il corrige. Celui-ci sera alors surligné en vert pour indiquer qu´une correction a été effectuée.\\
  Il est possible que le dictionnaire ne propose pas de correction satisfaisante. Il est toujours possible d´éditer à la main n´importe quel mot, via un clic sur ce dernier. Une fois la correction effectuée, et quelle qu´ait été la couleur du surlignage avant éditage, il apparaît vert.}
\end{itemize}
