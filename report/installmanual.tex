Nous allons détailler les manipulations nécessaires pour installer correctement KibiOCR.
\begin{itemize}
  \item{Pour utiliser notre projet, commencez par décompresser la tarball: ¨tar -xvjf corre_m-proj20131209.tar.bz2¨.}
  \item{Compilez KibiOCR en allant dans le dossier de la tarball extraite, et faites ¨make¨}
  \item{Installez un serveur web si vous n´en disposez pas déjà un sur votre machine. Nous avons utilisé Nginx pour sa simplicité, son efficacité et sa légèreté. Ce serveur est disponnible sur toutes ou presque les distributions Unix/Linux. Référez-en vous à la procédure propre à la votre.}
  \item{Copiez le contenu du dossier /src/guy dans le dossier du site.}
  \item{Ajoutez le binaire ¨kibi.native¨ dans le path de votre système, ou ajoutez un alias dans votre bashrc.}
  \item{Démarrez le serveur web.}
  \item{Vous pouvez maintenant utiliser le programme en vous connectant à votre addresse locale avec votre navigateur.}
\end{itemize}
