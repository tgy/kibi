\documentclass[12pt]{report}

\usepackage[top=1.2in, bottom=1.2in, left=1.4in, right=1.2in]{geometry}
\usepackage{graphicx}

\usepackage[T1]{fontenc}

\usepackage[francais]{babel}

\usepackage{fontspec}
\setmainfont{Droid Serif}
\setmonofont{Inconsolata}

\usepackage{url}
\usepackage{hyperref}
\usepackage{fancyhdr}
\pagestyle{fancy}

\usepackage{minted}

\linespread{1.3}

\definecolor{gray}{RGB}{155,155,155}
\definecolor{toogyblue}{RGB}{87,102,181}

\usepackage{indentfirst}

\newminted{ocaml}{bgcolor=gray!25,
                  fontsize=\normalsize,
                  mathescape,
                  linenos,
                  frame=lines,
                  framerule=0.3pt
                 }

\usepackage{titlesec, blindtext}
\definecolor{gray75}{gray}{0.75}
\newcommand{\hsp}{\hspace{20pt}}
\titleformat{\chapter}[hang]{\Huge\bfseries}{\thechapter\hsp\textcolor{gray75}{|}\hsp}{0pt}{\Huge\bfseries}

\fancyhf{}
\rhead{\textbf{\thepage}}
\lhead{\leftmark}
\rfoot{\leftmark}
\lfoot{\textbf{\thepage}}
\renewcommand{\headrulewidth}{0.4pt}
\renewcommand{\footrulewidth}{0.4pt}

\setcounter{tocdepth}{1}


\title{\includegraphics[scale=0.55]{chapters/Pictures/kibi.png}}


\author{Mathieu `RustyCrowbar' Corre \\
        Thibaud `zehir' Michaud \\
        Valentin `toogy' Iovene \\
        Pierre `Grimpow' Gorjux
      }

\date{11 décembre 2013}

\begin{document}

\maketitle

\tableofcontents

\chapter*{Introduction}

This document will present the current state of our project, KiBiOCR, as part of
EPITA's syllabus. KiBiOCR's name speaks for itself. It is an OCR (Optical
Character Recognition) program. Its goal is then to recognize characters in a
given image.\\

You will see that our project is breaked up in 5 very distinct parts:

\begin{itemize}
    \item{Lenna}: preprocessing part of the program. Lenna modifies user's input
        image to help other parts of the program to do their job.
    \item{Freddy}: `he' is in charge of the segmentation part, which is
        identifying images, paragraphs, words and characters positions.
    \item{Anna}: this is our artificial neural network (ANN). `She' identifies
        characters segmented by Freddy.
    \item{\emph{(Le petit)} Robert}: this is our dictionary. `He' detects the
        document language and errors made by Anna/Freddy in the identification
        process and then try to find the nearest words.
    \item{Guy}: our user interface. This is the way our program and the user
        will be able to interact.
\end{itemize}

We gave surnames to the parts of our program because we like private jokes but
mainly because shortnames are really convenient when we talk about our project.

% \chapter{The team}
% 
% \section{Valentin `toogy' Iovene}
% 
% \begin{wrapfigure}{l}{0.1\textwidth}
%     \vspace{-1cm}
%     \begin{center}
%         \includegraphics[width=0.1\textwidth]{images/toogy.png}
%     \end{center}
%     \vspace{-1cm}
% \end{wrapfigure}

\chapter{Lenna -- \emph{Preprocessing}}
Avant d'essayer de détecter les paragraphes, les lignes ainsi que les caractères dans l'image, et de reconnaître les caractères, il faut "nettoyer" l'image. Dans ce chapitre nous allons décrire différents algorithmes que nous avons implémentés dans ce but.\\

\section{Binarisation}
C'est le premier algorithme que nous avons écrit parce qu'il est simple, essentiel et c'est un bon point de départ pour se familiariser avec la bibliothèque et ses fonctions de base. Nous avons d'abord implémenté la méthode la plus naïve : calculer la moyenne des composantes RGB de chaque pixel pour avoir sa luminosité, et appliquer un seuil. Tout pixel dont la luminosité est en dessous de 127 est considéré comme blanc et les autres comme noir. Evidemment, ce ne fut pas notre algorithme définitif.\\

Nous aurions pu l'améliorer en prenant comme seuil la luminosité moyenne des pixels, cependant cela n'aurais pas été beaucoup mieux. Nous avons plutot décidé de rechercher un meilleur algorithme, et avons décidé que la méthode d'Otsu était la plus adaptée à notre cas. De plus cet algorithme est relativement simple à impémenter. La principe de la méthode d'Otsu est de trouver le seuil idéal pour l'algorithme décrit précedemment. Le seui idéal est défini comme étant celui qui minimise la variance entre les deux classes formées par ce seuil. Il faut donc essayer chaque seuil possible et calculer à chaque fois la variance de chaque classe. Heureusement une petite astuce mathématique nous permet de réduire de façon significative la complexité de cet algorithme en maximisant la variance inter-classe au lieu de minimiser la variance intra-classe.\\

\section{Réduction du bruit}

Nous avons éssayé deux méthodes pour réduire le bruit dans l'image : le flou gaussien et le filtre médian. Nous avons été insatisfait des deux car les lettres étaient moins lisible après le filtre. Pour corriger ça, nous avons réduit le nombre de pixel voisins pris en compte dans le filtre median : pour chaque pixel, nous prenons le pixel median parmi le pixel lui même et ses quatre voisins immédiats.\\

\section{Rotation}
\section{Algorithme de base}

Il est très difficile de détecter les zones de texte dans l'image si elle n'est pas droite. Nous avons donc des algorithme pour détecter l'angle du texte dans l'image, puis pour tourner l'image selon cet angle.\\

Une implémentation naïve commencerait par parcourir l'image d'origine et de transposer chaque pixel dans la nouvelle image grâce à une matrice de rotation. Cependant, cet algorithme donnera de très mauvais résultats puisque tous les pixels dans l'image crée n'aurons pas d'antécédants dans l'image d'origine, créant ainsi un effet d'aliasing.

A la place, nous effectuons l'opération inverse : pour chaque pixel de la nouvelle image (encore vide), il faut trouver le pixel correspondant dans l'image d'origine. Il suffit pour cela d'appliquer la matrice de rotation d'angle opposé.

A cause de l'imprecision des coordonnées entières, l'image de sortie n'est pas parfaite. C'est pourquoi nous avons utilisé un algorithme d'interpolation bilinéaire pour conserver la forme des caractères dans l'image avant de commencer la segmentation et l'identification des caractères.

\subsection{Interpolation Bilinéaire}
\subsection{Detection de l'angle}
\subsubsection{Transformée de Hough}

La transformée de Hough est une méthode générale pour trouver des lignes et des ellipses dans une image. Elle est souvent utilisée car une fois le principe mathématique compris, elle est simple à implémenter, donne de bons résultats et à une complexité assez faible par rapport, par exemple, à la transformée de Fourier. Nous n'avons besoin que du cas le plus simple de l'algorithme qui consiste à trouver les lignes dans une image binarisée.

La transformée de Hough se base sur une représentation très particulière des droites du plan : plutot que deux les représenter avec les deux paramètres usuels $a$ et $b$ dans l'équation $y = ax + b$, une droite est décrite comme un couple $(r, \theta)$ où $r$ est la distance de la ligne à l'origine et $\theta$ est l'angle de cette distance par rapport à l'abscisse.

En quoi cette représentation est-elle utile? Cela vient du fait que, étant donné un point de coordonnées $(x, y)$ et l'angle $\theta$ de la droite passant par ce point, il est facile d'exprimer la distance $r$ entre la ligne et l'origine. L'équation est la suivante :\\
$r = x\cos\theta + y\sin\theta$\\
Donc pour un point de coordonnées $(x, y)$ on peut tracer le graphe de $r$ en fonction de $\theta$. Il est facile de deviner d'après l'équation que ce graphe sera une sinusoïde. Cette sinusoïde représente l'ensemble des droites passant par ce point. Si nous stockons ce graphe dans un accumulateur, et ajoutons dans cet accumulateur la sinusoïde correspondant à tous les autres points, on obtient un graphe tableau dans lequel la valeur la plus grande représentera les coordonnées de la droite (dans l'espace de Hough) passant par le plus de points possible.

\subsubsection{More preprocessing}

Il y'a encore un problème à régler avant de pouvoir appliquer la transformée de Hough à notre image scannée : Une ligne de texte n'apparait pas comme une ligne droite. Nous avons essayé d'appliquer directement à une image binarisée, mais la plupart du temps il trouvait ou bien la diagonale de l'image, ou bien une reliure particulièrement visible. Après quelques recherches, nous avons trouvé une solution : on détecte tout d'abord les blocs de pixels noirs connexes, qui ne sont pas toujours mais très souvent des caractères, et on les remplaces par leur point centrale. L'utilité est triple : rendre plus apparentes (du point de vue de la transformée de Hough) les lignes de texte, réduire considérablement la complexité de l'algorithme et reduire l'importance des gros blocs de pixels comme les tâches ou les reliures.

This method has a major flaw that we wish to correct before the end of the
project: when the noise isn't reduced enough, every separate pixel in the
"noisy zone" is considered a block and gets a huge role in the Hough transform,
potentially leading the angle detection to fail completely. Every further step
of the OCR relies on a perfectly straight image, and it is therefore very
important that we correct this. \\


\chapter{Freddy - \emph{Segmentation}}

Freddy is our segmentation module. It uses OCSFML.
To detect lines as today, we first assume that all the previous preprocessing steps were done perfectly, which means no artefacts, good rotation, binarization... For now, we can only get text, in one column, and no images. We will work on those situations, and Freddy will be working as expected on time.

Freddy est notre module de segmentation, il utilise OCSFML.

\section{L'ancienne méthode, ses points faibles}

Au moment de la première soutenance, nous utilisions un algorithme très simple, et moyennement efficace: nous faisions un histogramme horizontal, un tableau de taille la hauteur en pixels de l´image. Nous le remplissions avec le nombre de pixels noirs à la ligne correspondante. Cette opération nous permettait de savoir à quel moment une nouvelle ligne ou un nouveau paragraphe apparaissent.\\
Une fois ce découpage effectué, il nous suffisait de répéter cette opération horizontalement sur chacune des lignes repérées, afin de détecter les mots et caractères composant chacune des lignes.\\
Cependant, nous avons été confrontés à plusieurs problèmes, qui nous ont amenés à chercher une autre méthode plus performante, que nous verrons en détail dans la prochaine section. En effet, avec cette méthode, les caractères avaient tendance à être coupés, ou bien collés à d´autres. Il en était de même pour les lignes, qui avaient tendance à fusionner. Cette technique nécessitait également d´avoir une image sortie du prétraitement parfaite, la moindre impureté ou la moindre ligne, image ruinant totalement le résultat de l´algorithme.

\section{Le RLSA}

Pour détecter les paragraphes, lignes, mots ou les caractères, nous utilisons donc maintenant l'algorithme Run Lenght Smooth Algorithm (RLSA) modifié par nos soins, pour couvrir au mieux nos besoins. Le principe est simple: on regarde horizontalement et verticalement quels sont les pixels noirs adjacents, c´est à dire quels sont les pixels noirs séparés de moins de n pixels noirs. Il en résultera donc deux matrices de bouléens aux dimensions de l´image. Dans ces matrices représentant la couleur noire ou blanche des pixels de l´image, nous aurons changé tous les pixels blancs entre deux pixels noirs adjacents. Les images représentées par ces matrices ressemblent à l´image d´entrée, mais comme si l´encre avait bavée horizontalement ou verticalement.\\
Nous appliquons ensuite un opérateur binaire logique entre les bits des deux matrices aux mêmes coordonées, afin de former une nouvelle matrice. Cette nouvelle matrice, toujours aux dimensions de l´image d´origine, représente l´image résultat, qui sera utilisée pour la suite des traitements. Dans cette image, les lettres, mots, lignes ou paragraphes se retrouveront rassemblés en blocs noirs, suivant le nombre de pixels adjacents que l´on aura considéré.\\
Cette transformation nous permet ensuite, via un autre algorithme que nous détaillerons plus tard, de récupérer les coordonées et dimensions des zones noircies.

\subsection{Optimisations}

Pour sélectionner les caractères, mots, lignes ou paragraphes, nous devons modifier la constante représentant le nombre de pixels adjacents à fusionner. Cependant, cette modification seule ne nous permet pas de sélectionner efficacement une partie précise de l´image. En effet, les caractères sont globalement carrés, tandis que les mots et lignes sont assez horizontaux. Une seule constante ne suffira donc

\section{Bounding boxes}

\begin{center}
\end{center}

\begin{center}
\end{center}

\section{Line and character detection}

\subsection{Making horizontal/vertical histogram}

To detect lines, we need to make histograms, containing the number of black pixels in every line/column of line of pixel of the provided image.We put everything in lists.

\subsection{Delimit lines/characters}

When the number of black pixel changes a lot, we decide to create or to end a line/to create or end a character. We put it in lists.

\subsection{Delimit lines/characters}

When the number of black pixel changes a lot, we decide to create or to end a line/to create or end a character. We put it in lists


\chapter{Anna -- \emph{Identification}}

After we have preprocessed the image and found the paragraphs, lines and characters in the document, we need to recognize these characters. To do this, we use a machine learning technique called a feed-forward neural network. In this chapter, we will describe in detail the result of our researches on the subject.

\section{Machine learning}

Machine learning is a field of research in which we try to design "agents" able
to improve their performance on a given task with experience. In the case of an
OCR software, the agent's task will be to classify images of characters, and the
experience will consist of feeding the agent with many examples. The examples
will be pairs of the form (image of the character, corresponding character).

\section{Univariate linear regression}

Lets look at a simpler problem: we have examples of the form $(x, y)$ where $x$
and $y$ are numbers, and our goal is to find a straight line to fit the
examples.\\

\begin{center}
\end{center}

The line will be of the form $h_w(x) = w_1x + w_0$. In order to determine the
most fitting line, we first need to define an error function. The sum of the
squared error appears to have nice properties (as being positive and
differentiable). This error function is expressed as follows:
$Loss(h_w) = \sum\limits_{j=1}^{N}(y_j - h_w(x_j))^{2}$. Now we need to minimize
this. To understand how we do this, it helps to visualize the graph ofthe Loss
in function of $w$.\\

Now imagine we start at a random point on this graph (i.e., we start with a
random line). To get to the minimum of the function, we only need to follow the
slope downward. Mathematically, this translates as substracting the gradient of
the Loss function from the current point, since the gradient points toward the
steepest slope upward. We usually multiply the gradient by a constant called the
learning rate, noted $\alpha$, to regulate the size of the steps. A slow
learning rate means converging slowly but a large learning rate might lead to
stepping over the minimum. The update rule is then:\\

$w_i \leftarrow w_i - \alpha \frac{\partial}{\partial w_i} Loss(w)$\\

We won't compute the derivative right now as we don't need it.

\section{Multivariate linear regression}

The extension to a multivariate problem is pretty straight forward. Lets denote
by $\textbf{x} = (x_0, x_1, ..., x_n)$ the input vector. The estimation function
then becomes $h_w(\textbf{x}) = w_nx_n + ... + w_1x_1 + w_0x_0$.

\section{Logistic regression}

What we want is a classification algorithm. We only need to slightly modify the
previous algorithm to make it a binary classifier: we keep the estimation
function $h_w(x)$, but we then feed the result to another function whose output
is in $[0;1]$. One such function is the step function, which worth 0 if $h_w(x)$
is negative and 1 otherwise. But this function is not differentiable, and we
will see later that it is a problem. So we usually prefer the logistic function,
or sigmoid function:\\

\begin{center}
\end{center}

Its equation is $g(x) = \frac{1}{1 + e^{-x}}$, and its derivative is
$g(x)(1 - g(x))$. Applying the same algorithm as for the multivariate linear
regression, the update rule becomes:\\

$w_i \leftarrow w_i + \alpha (y - h_w(\textbf{x}))
h_w(\textbf{x})(1 - h_w(\textbf{x}))x_i$\\

There is no need here to explain the derivation of the Loss function.

\section{Perceptron}

The logistic regression function will be the elementary brick for our neural
network. It is what we will call a neuron, since it is very close to the real
principle of biological neuron. Now lets complexify the problem a little bit: we
need to classify the input data between more than two possible outcomes. Let's
say for example that we need three possible outcomes. We only need to connect
the input to three different logistic regression units. This will produce an
output vector instead of a single value, and ideally this vector will converge
toward something like (1, 0, 0), (0, 1, 0) or (0, 0, 1), each time representing
one of the three possible cassifications. This is called a single-layer
feed-forward neural network, or perceptron.\\

\begin{center}
\end{center}

\section{multi-layer feed-forward neural network}

A feed-forward neural network is a perceptron, whose output will be the input of
another perceptron, etc. until it reaches the output layer. Every layer that is
not the input layer or the output layer is called a hidden layer. The big
advantage of a multi-layer feed-forward neural network over a perceptron is that
it can potentially compute any continuous function with only one hidden layer
and with the right number of neurons in each layer, according to the universal
approximation theorem.\\

\begin{center}
\end{center}

What changes between the perceptron and the multi-layer feed-forward neural
network is the derivative of the loss function. Once again, there is no need to
enter into the details of the maths behind the derivation, but the only thing we
need to know when implementing the neural network is that the derivative of the
parameters in one layer depends on the derivative of the parameters of the next
layer. Therefore we need to compute the derivative in the output layer, then the
hidden layer, hence the term "backpropagation".

\section{Optimizations}

The only optimization we implemented at this point is the momentum. In the
gradient descent process, we can consider that the current point has some
inertia by adding a certain percentage of the gradient of the last iteration in
the update rule. There are two purposes to this: converging faster, and avoiding
local minimum.


\chapter{Robert -- \emph{Spell checking}}

Robert is a module, which can detect the language used in a given text, and eventually indicate
to the user which word is wrong, and give a selection of possible corrections.

\section{Generation of Dictionnaries}
A dictionnary is a file which contains a lot of words of a specific language we can find in
a text.\\
We didn't create them ; it's too long and we can easily find some complete dictionnaries on
Internet (~150 000 words in English, ~700 000 words in French, for example).\\
But actually, searching a word in a file is'nt a really good idea and may take a long time.
That's why before doing anything else, we have to generate a data structure which will store
the whole words of the dictionnary file. This one is a hashtable, because the
time taken to search a common word is constant.\\
So, when loading Robert, OCaml will generate the whole dictionnaries needed by creating hashtables
and put them in a list.\\
\\
But there is still a problem : the time taked to create a hashtable from a dictionnary is around 1
second. If we want to accept many differents languages, the loading of the module would be too long.
The solution is the Marshall module : it let us serialize an object into a bytes array, and save it
into files. The opposite procedure is also possible.
So OCaml has juste to read the byte array and deserialize it to load a hashtable. This technic need
more memory but is really faster (around 1000\% faster)
\section{Detecting language in a given Text}
The algorithm used is pretty simple: it will analyse the first X words (X is an empirical number, set
as 200). 
For each language, it will count the number of words recognised ; The final language is the one which has the mosts of right matches.
\section{Detection of wrong words}
It's quite the same thing. The algorithm make a list, and starts to read the whole text. Each time a word is wrong, it is added to the list.
\section{Selection of possible corrections for a wrong word}
Generally, when a word is wrong, it's because the user writted it thinking about a right word, with the same pronunciation.\\
That's why the Soundex algorithm has been used in our project. It transforms a given string into its phonetic equivalent.
So, we created a new type of dictionnary : a "phonetic hashtable" : we can find the word by searching the phonetic string.
So, if we're searching a phonetic string we can find the whole words which have the same phonetic string.
We have a first list with a lot of possible corrections.\\
But we won't give for examples 42 possibilies of correction to a wrong word: the user would be lost ; that's why we'll give
to the user only 5 possibilities.\\
The possibilities will be sorted using another algorithm, called Levenshtein.
Levenshtein calculates the number of modifications we have to make in order to change a word A into a word B.
So, we'll only have the 5 words which are the closest to the wrong word. 


\chapter{Guy - \emph{Graphical User Interface}}

\section{A quick overview of Guy}

Guy is our GUI. It uses LablGTK and OCSFML, and will probably be using html to render the text output. It has several roles that we will describe right now. It is today composed of one window, to choose the file. It will be, for the second soutenance, composed of two or three more windows, to show the flow of the processing and the time taken by every main algorithm that we are using. 

\begin{center}
\end{center}

\section{A user-friendly interface}

\subsection{Helping the user to choose the image to process}

Guy has a built-in file browser, which is very intuitive, and easy to use, to select the file you want to use. Once the file has been selected, it's path will be displayed in an editable text box. However, the user can directly type in the provided text box the path of his desired file. Of course, it is still possible to use our program via the command-lines.

\subsection{Displaying the usefull informations}

We will display a miniature image of the image (assuming that it really is an image) that the user choosed.

\subsection{Showing the final result to the user}

Guy will be charged to show to the user the final work of our OCR. Indeed, we will present the image the user gave us, and the text we could detect, with the eventual orthograph corrections using Robert. It needs to be well organised, so that in case of a long document, the user could find where everything in the output text is located in his image, and vice-versa.

\section{Guy is the foreman}

One of the role of our GUI is to make every module work one after the other, calling them one after the other, with the right parameters. It will be possible to include one or several progress bars to inform the user of the flow of the processing.

 
\chapter{Website}

\section{Python}

\begin{center}
\end{center}

This year, we decided to switch from PHP to Python. We think that Python is a
stronger and more relevant language and we wanted to learn its syntax and its
specifics. So, here we are. \\

Plus, if we want to put our OCR online so that people can use it via a web
interface, it would me so much easier to do it with Python than with PHP. Python
can easily call other programs and execute commands on the server in a very
secure way. It is more complicated with PHP.

\section{Django}

\begin{center}
\end{center}

There is a very nice Python web framework out there on the web that is called
Django. Working with Django is quick: it is made for rapid development and for
perfectionists (we are).

\section{Nginx}

\begin{center}
\end{center}

We wanted to use something else than a basic Apache server so we decided to go
with Nginx, a quick, clean and easy to configure web server.

\section{Github}

\begin{center}
\end{center}

Our code is hosted on the now very famous website (and git server) GitHub. For
the moment, it is located in a private repository (for obvious reasons) but we
plan to go opensource once the last soutenance will be passed!


\chapter*{Conclusion}

Our project is moving forward very well. Our team is organized, we are all
working with passion and seriousness. We often schedule meetings and establish
task lists so that everyone is aware of what has to be done and can pick up
something he is interested in. \\

Some very tough tasks are waiting for us and we hope we will be able to show up
at the final with a fresh and working piece of art.

\end{document}


